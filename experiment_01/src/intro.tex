% Notes
%% - Introduction to the experiment
%% - Purpose of the experiment
%% - Background information
%% - Hypothesis

%% Used Polaroid polarizers (which produce linearly polarized light), a more-or-less collimated bright white light source, 
%% and an intensity meter (photometer) to understand how concentration of sugar in water affects the polarization of light.


In order to understand how sugar concentration in water affects the polarization of light, we measured the intensity of light exiting a second, 
constant angle polarizer as a function of the angle of the first polarizer. We used a bright white light source, a photometer, 
and a series of beakers varying in size with different concentrations of sugar in water. By using two polarizers, one at a constant angle and the other at varying angles, we were able to determine
the relative polarization change by the sugar water. For each trial, the first (varying) polarizer began at 90 degrees to the second (constant) polarizer, 
and was then rotated in 10 degree increments to 0 degrees relative to the second polarizer. We first began by measuring the intensity of light exiting the second polarizer for no beaker, 
then small, medium, and large beakers with no water. This initial dataset provided us with a baseline for the intensity of light exiting the second polarizer.