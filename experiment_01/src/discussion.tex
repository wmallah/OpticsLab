% Methods for analysis of data

The first step in processing the raw data was to investigate the shape of intensity versus polarization curves for each different solution. \Cref{no_water_figure,water_figure,solution1_figure,solution2_figure,solution3_figure,solution4_figure} show this relationship for no water, water, and solutions 1-4. Each of these graphs were created using the "matplotlib" package in Python. Uncertainties and error propogation were managed using the "uncertainties" Python package. 
The next step was to fit each of these curves to a generic cosine function to determine the shift in phase. This process was accomplished by fitting the intensity versus polarization data to a cosine function in the form $A\cos{\left(Bx+C\right)}+D$. We invoked the ODR (orthogonal distance regression) class from the "SciPy" Python package to fit these data sets to the previously mentioned cosine function.
After fitting each of the curves to a cosine function, the phase shift for each fit was extracted and plotted against concentration in Figures 8-10 for each beaker size.
Finally, the phase shift was plotted against concentration times path length as an all-encompassing figure to determine if the polarization changes linearly with respect to both concentration and path length.